
\documentclass{article} % Especially this!
% _____        _____ _  __          _____ ______  _____ 
%|  __ \ /\   / ____| |/ /    /\   / ____|  ____|/ ____|
%| |__) /  \ | |    | ' /    /  \ | |  __| |__  | (___  
%|  ___/ /\ \| |    |  <    / /\ \| | |_ |  __|  \___ \ 
%| |  / ____ \ |____| . \  / ____ \ |__| | |____ ____) |
%|_| /_/    \_\_____|_|\_\/_/    \_\_____|______|_____/ 
%%%%%%%%%%%%%%%%%%%%%%%%%%%%%%%%%%%%%%%%%%%%%%%%%%%%%%%%

\usepackage[english]{babel}
\usepackage[utf8]{inputenc}
\usepackage[margin=1.5in]{geometry}
\usepackage{amsmath}
\usepackage{amsthm}
\usepackage{amsfonts}
\usepackage{amssymb}
\usepackage[usenames,dvipsnames]{xcolor}
\usepackage{graphicx}
\usepackage[siunitx]{circuitikz}
\usepackage{tikz}
\usepackage[colorinlistoftodos, color=orange!50]{todonotes}
\usepackage{hyperref}
\usepackage[numbers, square]{natbib}
\usepackage{fancybox}
\usepackage{epsfig}
\usepackage{soul}
\usepackage[framemethod=tikz]{mdframed}
\usepackage[shortlabels]{enumitem}
\usepackage[version=4]{mhchem}
\usepackage{listings}



%%%%%%%%%%%%%%%%%%%%%%%%%%%%%%%%%%%%%%%%%%%%%%%%%%%%%%%


%   _____ _    _  _____ _______ ____  __  __ 
%  / ____| |  | |/ ____|__   __/ __ \|  \/  |
% | |    | |  | | (___    | | | |  | | \  / |
% | |    | |  | |\___ \   | | | |  | | |\/| |
% | |____| |__| |____) |  | | | |__| | |  | |
%  \_____|\____/|_____/   |_|  \____/|_|  |_|
%%%%%%%%%%%%%%%%%%%%%%%%%%%%%%%%%%%%%%%%%%%%%%%%
%  _____ ____  __  __ __  __          _   _ _____   _____ 
% / ____/ __ \|  \/  |  \/  |   /\   | \ | |  __ \ / ____|
%| |   | |  | | \  / | \  / |  /  \  |  \| | |  | | (___  
%| |   | |  | | |\/| | |\/| | / /\ \ | . ` | |  | |\___ \ 
%| |___| |__| | |  | | |  | |/ ____ \| |\  | |__| |____) |
% \_____\____/|_|  |_|_|  |_/_/    \_\_| \_|_____/|_____/ 
%%%%%%%%%%%%%%%%%%%%%%%%%%%%%%%%%%%%%%%%%%%%%%%%%%%%%%%%%%

% SYNTAX FOR NEW COMMANDS:
%\newcommand{\new}{Old command or text}

% EXAMPLE:

\newcommand{\blah}{blah blah blah \dots}

%%%%%%%%%%%%%%%%%%%%%%%%%%%%%%%%%%%%%%%%%%%%%%%%%%%%%%%%%
%  _______ ______          _____ _    _ ______ _____  	%
% |__   __|  ____|   /\   / ____| |  | |  ____|  __ \ 	%
%    | |  | |__     /  \ | |    | |__| | |__  | |__) |	%
%    | |  |  __|   / /\ \| |    |  __  |  __| |  _  / 	%
%    | |  | |____ / ____ \ |____| |  | | |____| | \ \ 	%
%    |_|  |______/_/    \_\_____|_|  |_|______|_|  \_\	%
%%%%%%%%%%%%%%%%%%%%%%%%%%%%%%%%%%%%%%%%%%%%%%%%%%%%%%%%%
%														%
% 			COMMANDS				SUMMARY				%
% \clarity{points}{comment} >>> "Clarity of Writing"	%
% \other{points}{comment}	>>> "Other"					%
% \spelling{comment}		>>> "Spelling"				%
% \units{comment}			>>> "Units"					%
% \english{comment}			>>> "Syntax and Grammer"	%
% \source{comment}			>>> "Sources"				%
% \concept{comment}			>>> "Concept"				%
% \missing{comment}			>>> "Missing Content"		%
% \maths{comment}			>>> "Math"					%
% \terms{comment}			>>> "Science Terms"			%
%														%
%%%%%%%%%%%%%%%%%%%%%%%%%%%%%%%%%%%%%%%%%%%%%%%%%%%%%%%%%
\setlength{\marginparwidth}{3.4cm}


% NEW COUNTERS
\newcounter{points}
\setcounter{points}{100}
\newcounter{spelling}
\newcounter{english}
\newcounter{units}
\newcounter{other}
\newcounter{source}
\newcounter{concept}
\newcounter{missing}
\newcounter{math}
\newcounter{terms}
\newcounter{clarity}

% COMMANDS

\definecolor{myblue}{rgb}{0.668, 0.805, 0.929}
\newcommand{\hlb}[2][myblue]{ {\sethlcolor{#1} \hl{#2}} }

\newcommand{\clarity}[2]{\todo[color=CornflowerBlue!50]{CLARITY of WRITING(#1) #2}\addtocounter{points}{#1}
\addtocounter{clarity}{#1}}

\newcommand{\other}[2]{\todo{OTHER(#1) #2} \addtocounter{points}{#1} \addtocounter{other}{#1}}

\newcommand{\spelling}{\todo[color=CornflowerBlue!50]{SPELLING (-1)} \addtocounter{points}{-1}
\addtocounter{spelling}{-1}}
\newcommand{\units}{\todo{UNITS (-1)} \addtocounter{points}{-1}
\addtocounter{units}{-1}}

\newcommand{\english}{\todo[color=CornflowerBlue!50]{SYNTAX and GRAMMAR (-1)} \addtocounter{points}{-1}
\addtocounter{english}{-1}}

\newcommand{\source}{\todo{SOURCE(S) (-2)} \addtocounter{points}{-2}
\addtocounter{source}{-2}}
\newcommand{\concept}{\todo{CONCEPT (-2)} \addtocounter{points}{-2}
\addtocounter{concept}{-2}}

\newcommand{\missing}[2]{\todo{MISSING CONTENT (#1) #2} \addtocounter{points}{#1}
\addtocounter{missing}{#1}}

\newcommand{\maths}{\todo{MATH (-1)} \addtocounter{points}{-1}
\addtocounter{math}{-1}}
\newcommand{\terms}{\todo[color=CornflowerBlue!50]{SCIENCE TERMS (-1)} \addtocounter{points}{-1}
\addtocounter{terms}{-1}}


\newcommand{\summary}[1]{
\begin{mdframed}[nobreak=true]
\begin{minipage}{\textwidth}
\vspace{0.5cm}
\begin{center}
\Large{Grade Summary} \hrule 
\end{center} \vspace{0.5cm}
General Comments: #1

\vspace{0.5cm}
Possible Points \dotfill 100 \\
Points Lost (Science Terms) \dotfill \theterms \\
Points Lost (Syntax and Grammar) \dotfill \theenglish \\
Points Lost (Spelling) \dotfill \thespelling \\
Points Lost (Units) \dotfill \theunits \\
Points Lost (Math) \dotfill \themath \\
Points Lost (Sources) \dotfill \thesource \\
Points Lost (Concept) \dotfill \theconcept \\
Points Lost (Missing Content) \dotfill \themissing \\
Points Lost (Clarity of Writing) \dotfill \theclarity \\
Other \dotfill \theother \\[0.5cm]
\begin{center}
\large{\textbf{Grade:} \fbox{\thepoints}}
\end{center}
\end{minipage}
\end{mdframed}}

%#########################################################

%To use symbols for footnotes
\renewcommand*{\thefootnote}{\fnsymbol{footnote}}
%To change footnotes back to numbers uncomment the following line
%\renewcommand*{\thefootnote}{\arabic{footnote}}

% Enable this command to adjust line spacing for inline math equations.
% \everymath{\displaystyle}

% _______ _____ _______ _      ______ 
%|__   __|_   _|__   __| |    |  ____|
%   | |    | |    | |  | |    | |__   
%   | |    | |    | |  | |    |  __|  
%   | |   _| |_   | |  | |____| |____ 
%   |_|  |_____|  |_|  |______|______|
%%%%%%%%%%%%%%%%%%%%%%%%%%%%%%%%%%%%%%%

\title{
\normalfont \normalsize 
\textsc{Indian Institute of Technology Bombay \\ 
CS684 Autumn Semester 2016} \\
[10pt] 
\rule{\linewidth}{0.5pt} \\[6pt] 
\huge Initialization and GPIO \\
\rule{\linewidth}{2pt}  \\[10pt]
}
\author{E.R.T.S. Lab}
\date{\normalsize \today}

\begin{document}

\maketitle
\noindent
% Date Performed \dotfill December 31, 1999 \\
% Partners \dotfill Full Name \\
% Instructor \dotfill Instructor's Name \\

%%%%%%%%%%%%%%%%%%%%%%%%%%%%%%%%%%%%%%%

%			  ______      ____  
%			 |  ____/\   / __ \ 
%			 | |__ /  \ | |  | |
%			 |  __/ /\ \| |  | |
%			 | | / ____ \ |__| |
%			 |_|/_/    \_\___\_\
%%%%%%%%%%%%%%%%%%%%%%%%%%%%%%%%%%%%%%%%

%
% Ctrl + / to comment out a group of lines.
%
%
% LIST MORE COMMON COMMMANDS
% LIST USEFUL WEBSITES FOR TABLES, ETC
% WHAT TO DO WHEN YOUR CODE WONT COMPILE
% OVERLEAF SHORTCUTS
%



%%%%%%%%%%%%%%%%%%%%%%%%%%%%%%%%%%%%%%%


% _               ____  
%| |        /\   |  _ \ 
%| |       /  \  | |_) |
%| |      / /\ \ |  _ < 
%| |____ / ____ \| |_) |
%|______/_/    \_\____/ 
%%%%%%%%%%%%%%%%%%%%%%%%
%  _____ _______       _____ _______ _____ 
% / ____|__   __|/\   |  __ \__   __/ ____|
%| (___    | |  /  \  | |__) | | | | (___  
% \___ \   | | / /\ \ |  _  /  | |  \___ \ 
% ____) |  | |/ ____ \| | \ \  | |  ____) |
%|_____/   |_/_/    \_\_|  \_\ |_| |_____/ 
%%%%%%%%%%%%%%%%%%%%%%%%%%%%%%%%%%%%%%%%%%%
% _    _ ______ _____  ______ 
%| |  | |  ____|  __ \|  ____|
%| |__| | |__  | |__) | |__   
%|  __  |  __| |  _  /|  __|  
%| |  | | |____| | \ \| |____ 
%|_|  |_|______|_|  \_\______|
%%%%%%%%%%%%%%%%%%%%%%%%%%%%%%
\section{Lab Objective}
% Yada Yada Yada
\begin{enumerate}
\item 
Understand IO operation in TMS4C123GXL
\item
Get acquainted with using on-board RGB LED and User Switches
\item
Generating delay using SysCtlDelay()
\end{enumerate}






%%% Pre-requisite
\section{Pre-requisite}
This lab assumes you have completed Lab-0, which means you are aware of creating new project in
CCS, making required configurations, technique to load and run user written program on the board
and you have run the demo code of LED blink (given in Lab-0).


%%% Problem Statement
\section {Problem Statement}

% Materials go here
In this lab you have to use switch SW1, SW2 and RGB LED present on Tiva C series board. You
have to create a new project (instructions for project creation in lab-0 handout) and use lab-1.c file.
\begin{enumerate}
\item
Use switch SW1 to Turn on Red LED on first switch press, Green LED on second switch
press and Blue LED on third switch press. Repeat the same cycle next switch press onwards.
Note that LED should remain on for the duration switch is kept pressed i.e. LED should turn
off when switch is released. Show the result to TA.
\item
Use switch SW2 and sw2Status (a variable). Your program should increment sw2Status by
one, every time switch is pressed. Note how the value of sw2Status changes on each switch
press. Use debugger and add sw2Status to “Watch Expression” window. Does the value of
sw2Status increment by one always? Show the result to TA.
Note: Define sw2Status as a global variable and in debug perspective use continuous refresh option
(You will find Continuous Refresh button on top of the Expression Window). You can use step debugging
or breakpoints to check the variable value.
\\
\\
\textbf{Hint}:\textit{To add variable to Expression Window, select and right click the variable name and select “Add
Watch Expression”. To view Expression Window, click on View button from CCS menu bar and
select Expressions.}
\item
Configure SW1 and SW2 such that:
\\
\hspace{2mm}Every time SW1 is pressed toggle delay of LED should cycle through
approximately 0.5s, 1s, 2s (Of any one color).
\\
\hspace{2mm}Every time SW2 is pressed color of LED should cycle through Red, Green
and Blue.

\end{enumerate}
%%%%%%%%%%%%%%%%%%%%%%%
% FOR A NUMBERED LIST
% \begin{enumerate}
% \item Your_Item
% \end{enumerate}
%%%%%%%%%%%%%%%%%%%%%%%
% FOR A BULLETED LIST
% \begin{itemize}
% \item Your_Item
% \end{itemize}
%%%%%%%%%%%%%%%%%%%%%%%








%#################################################################
% _____  _____   ____   _____ ______ _____  _    _ _____  ______ 
%|  __ \|  __ \ / __ \ / ____|  ____|  __ \| |  | |  __ \|  ____|
%| |__) | |__) | |  | | |    | |__  | |  | | |  | | |__) | |__   
%|  ___/|  _  /| |  | | |    |  __| | |  | | |  | |  _  /|  __|  
%| |    | | \ \| |__| | |____| |____| |__| | |__| | | \ \| |____ 
%|_|    |_|  \_\\____/ \_____|______|_____/ \____/|_|  \_\______|
%%%%%%%%%%%%%%%%%%%%%%%%%%%%%%%%%%%%%%%%%%%%%%%%%%%%%%%%%%%%%%%%%%%
\section {Relevant Theory}
%%%%%%%%%%%%%%%%%%%%%%%
% FOR A NUMBERED LIST
% \begin{enumerate}
% \item Your_Item
% \end{enumerate}
%%%%%%%%%%%%%%%%%%%%%%%

\begin{enumerate}
\item 
Read \href{https://www.cse.iitb.ac.in/~erts/html_pages/Resources/Tiva/TM4C123G_LaunchPad_Workshop_Workbook.pdf}{Resource7} available on course web page -- Texas Instrument “TM4C123G
LaunchPad Workshop - Student Guide and Lab Manual”. You should go through Chapter-3
“Introduction to TivaWare, Initialization and GPIO” of the Manual.
\item
You will use TivaWare Peripheral Driver Library, an API written by Texas Instrument to
access different peripherals and functionality of ARM Cortex-M based micro controller. User
Guide for Peripheral Driver Library can be downloaded from course web page\href{https://www.cse.iitb.ac.in/~erts/html_pages/resources.html}{Resource4} .
\end{enumerate}








%%%Procedure
%%%%%%%%%%%%%%%%%%%%%%%%%%%%%%
\section {Procedure}
%%%%%%%%%%%%%%%%%%%%%%%%%%%%%%
% TO IMPORT AN IMAGE
% UPLOAD IT FIRST (HIT THE PROJECT BUTTON TO SHOW FILES)
% KEEP THE NAME SHORT WITH NO SPACES!
% TYPE THE FOLLOWING WITH THE NAME OF YOUR FILE
% DON'T INCLUDE THE FILE EXTENSION
% \includegraphics[width=\textwidth]{name_of_file}
% \textwidth makes the picture the width of the paragraphs
%%%%%%%%%%%%%%%%%%%%%%%%%%%%%%
% TO CREATE A FIGURE WITH A NUMBER AND CAPTION
% \begin{figure}
% \includegraphics[width=\textwidth]{image}
% \caption{Your Caption Goes Here}
% \label{your_label}
% \end{figure}
% REFER TO YOUR FIGURE LATER WITH
% \ref{your_label}
% LABELS NEED TO BE ONE WORD
%%%%%%%%%%%%%%%%%%%%%%%%%%%%%
\begin{enumerate}
\item 
Include all the relevant header files in your code. Ensure that the following header files are present:
\begin{lstlisting}
include "inc/hw_types.h"
include "inc/hw_memmap.h"
include "driverlib/sysctl.h"
include "driverlib/gpio.h"
include "inc/hw_ints.h"
include <time.h>
\end{lstlisting}
\item
Make sure you unlock Port F pins 0 and 4 to register SW2 and SW1 switch press. Please refer to section 3 of Resource 7 for a code sequence to unlock these pins.
\item
"SysCtlDelay(6700000)" generates a delay of about 500ms, use this to generate other delays.
\end{enumerate}
%%% Demo and Submissions
\section {Demo and Submissions}
%%%%%%%%%%%%%%%%%%%%%%%%%%%%%%%%%%%%%%%%%%%%%%%%
You have to shoot three individual videos demonstrating the output the problem statement.
Your codes for each of the problem statement has to be uploaded in Github repository.








%\subsection{Definitions}
% Include your sources!
%%%%%%%%%%%%%%%%%%%%%%%
% LIST OF DEFINITIONS
% \begin{description}
% \item [WORD] {Definition}
% \end{description}
%%%%%%%%%%%%%%%%%%%%%%%






%\subsection{Results}
% State your main discovery based on the experimental data.






%\subsection{Questions}
% Write full question and format answers in ITALIC
% CTRL + I for ITALIC











%  _____  ____  _    _ _____   _____ ______  _____ 
% / ____|/ __ \| |  | |  __ \ / ____|  ____|/ ____|
%| (___ | |  | | |  | | |__) | |    | |__  | (___  
% \___ \| |  | | |  | |  _  /| |    |  __|  \___ \ 
% ____) | |__| | |__| | | \ \| |____| |____ ____) |
%|_____/ \____/ \____/|_|  \_\\_____|______|_____/ 
%%%%%%%%%%%%%%%%%%%%%%%%%%%%%%%%%%%%%%%%%%%%%%%%%%%%


% USE NOCITE TO ADD SOURCES TO THE BIBLIOGRAPHY WITHOUT SPECIFICALLY CITING THEM IN THE DOCUMENT

%\nocite{ref_num}


%%%%%%%%%%%%%%%%%%%%%%%%%%%%%%%%%%%%%%%%%%%%%%%%%%%%%%

			% BIBLIOGRAPHY: %

% Make sure your class *.bib file is uploaded to this project by clicking the project button > add files. Change 'sample' below to the name of your file without the .bib extension.
%%%%%%%%%%%%%%%%%%%%%%%%%%%%%%%%%%%%%%%%%%%%%%%%%%

%\bibliographystyle{plainnat}
%\bibliography{sample}

% UNCOMMENT THE TWO LINES ABOVE TO ENABLE BIBLIOGRAPHY

%%%%%%%%%%%%%%%%%%%%%%%%%%%%%%%%%%%%%%%%%%%%%%%%%%


\end{document} % NOTHING AFTER THIS LINE IS PART OF THE DOCUMENT
